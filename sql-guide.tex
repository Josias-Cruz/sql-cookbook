\documentclass{article}
\usepackage{listings}
\usepackage{color}

\definecolor{dkgreen}{rgb}{0,0.6,0}
\definecolor{gray}{rgb}{0.5,0.5,0.5}
\definecolor{mauve}{rgb}{0.58,0,0.82}

\lstset{
  language=SQL,
  aboveskip=3mm,
  belowskip=3mm,
  showstringspaces=false,
  columns=flexible,
  basicstyle={\small\ttfamily},
  numbers=none,
  numberstyle=\tiny\color{gray},
  keywordstyle=\color{blue},
  commentstyle=\color{dkgreen},
  stringstyle=\color{mauve},
  breaklines=true,
  breakatwhitespace=true,
  tabsize=3
}

\begin{document}

\title{SQL Guide for Programmers with Business Cases}
\author{Josias Cruz}
\maketitle

\section{Introduction}
This guide provides an overview of SQL for programmers. It includes basic SQL syntax and several use cases. I'm writing this guide as a form of 
cookbok if you will. This guide will include basic 'ingredients' of a query plus basic subsets of queries that will handle most usecases.
It's meant to be read a recipe book for queries, with each snippet being its own individual set. I'm writing this guide mostly out of my own self need,
but also it's a living document that will be updated periodically as I learn more and grow. This guide assumes you have basic knowledge of cloud, and sql.
I will be referencing Google BigQuery public datasets so they'll be accessible for nearly everyone. It also helps that you'll be able to run SQL queries in BigQuery
as well, although if you use different databases and stacks it's fair game too. 

\section{SQL Basics}
SQL, or Structured Query Language, is a language designed to manage data in relational databases.



\subsection{SELECT Statement}
The SELECT statement is used to select data from a database. The data returned is stored in a result table, called the result-set.

\begin{lstlisting}
SELECT column1, column2, ...
FROM table_name;
\end{lstlisting}

\section{Use Cases}
In this section, we will go over some common use cases for SQL.

\subsection{Use Case 1}
Describe the first use case here.

\begin{lstlisting}
-- SQL code for use case 1
\end{lstlisting}

\subsection{Use Case 2}
Describe the second use case here.

\begin{lstlisting}
-- SQL code for use case 2
\end{lstlisting}

\end{document}